% Template for ICASSP-2015 paper; to be used with:
%          spconf.sty  - ICASSP/ICIP LaTeX style file, and
%          IEEEbib.bst - IEEE bibliography style file.
% --------------------------------------------------------------------------
\documentclass{article}
\usepackage{spconf,amsmath,graphicx}

% Example definitions.
% --------------------
\def\x{{\mathbf x}}
\def\L{{\cal L}}

% Title.
% ------
\title{Modulation Design in Amplify-and-Forward Two-Way Relay HARQ Channel}
%
% Single address.
% ---------------
\name{Author(s) Name(s)\thanks{Thanks to XYZ agency for funding.}}
\address{Author Affiliation(s)}
%
% For example:
% ------------
%\address{School\\
%	Department\\
%	Address}
%
% Two addresses (uncomment and modify for two-address case).
% ----------------------------------------------------------
%\twoauthors
%  {A. Author-one, B. Author-two\sthanks{Thanks to XYZ agency for funding.}}
%	{School A-B\\
%	Department A-B\\
%	Address A-B}
%  {C. Author-three, D. Author-four\sthanks{The fourth author performed the work
%	while at ...}}
%	{School C-D\\
%	Department C-D\\
%	Address C-D}
%
\begin{document}
%\ninept
%
\maketitle
%
\begin{abstract}
  % (what we do)
  As a practical transmission enhancement technique for relay and HARQ system,
  Modulation Diversity (MoDiv) uses distinct mappings to the same constellation
  for different (re)transmissions. 
  % (how they do)
  % (how we do)
  In this work, we study the MoDiv optimization
  in a Amplify-and-Forward (AF) Two-Way Relay Channel (TWRC). The design of
  MoDivto minimize the bit-error rate (BER) is formulated into a successive
  Koopmans-Beckmann Quadratic Assignment Problem (QAP), which is solved
  sequatially with a robust taboo search method.
  % (main results)
  The performance gain of our MoDiv scheme over retransmission without remapping
  and a heuristic MoDiv scheme is demonstrated with numerical ressults.
\end{abstract}
%
\begin{keywords}
  Modulation diversity, two-way relay, amplify-and-forward, HARQ, QAP
\end{keywords}
%
\section{Introduction}
\label{sec:intro}

% Motivation and literature review for Two-way relay-HARQ. HARQ first,
% then two-way relay
As an advanced technique to improve the robustness of high-rate wireless
transmissions against poor channel conditions, Hybrid Automatic Repeat reQuest
(HARQ) has found its application in various communication
systems~\cite{cripriano2010overview}. HARQ works on both PHY layer and MAC
sublayer to mitigate packet loss due to channel fading and
link-adaptation accuracy. Recently, substantial research interest
has been drawn to HARQ in Two-Way Relay Channel
(TWRC) [2-4].
In~\cite{iannello2009throughput}, the average throughput of naive Type-I HARQ
policy for both Amplify and Forward (AF) and Decode and Forward (RF) TWRC schemes have
been analyzed. The energy-delay tradeoff, and the diversity-multiplexing
tradeoff of type-II HARQ policy, also known as full Incremental Redundancy (IR),
for AF TWRC scheme have been studied in~\cite{choi2013energy}
and~\cite{xu2014diversity}, respectively. Related works about TWRC with ARQ
for different relay schemes and retransmission policies can also be
found in~\cite{popovski2007wireless, chen2012arq, guan2015twoway} and the references therein.

% Literature review on MoDiv for HARQ-CC
Apart from the naive Type-I HARQ and HARQ-IR, Type-I HARQ with maximal ratio
combining (MRC), also known as HARQ-Chase Combining
(HARQ-CC)~\cite{chase1985code}, is another simple and practical HARQ
scheme supported by such standands as HSPA~\cite{TS25.308},
LTE~\cite{sesia2009lte} and so forth. As practical transmissions often admit
linear modulations of finite-alphabet constellation (e.g. Q-ary QAM), the
performance of HARQ-CC can be improved with Modulation Diversity
(MoDiv)~\cite{benelli1992new}, in which a same group of $\log_2Q$ bits are
mapped to different symbols in a same constellation in different round of
(re)transmissions. MoDiv has been studied for
HARQ~\cite{harvind2005symbol}, relay
networks~\cite{seddik2008trans, khormuji2008rate} and relay-HARQ
systems~\cite{kim2009design, ryu2011ber}.

% What we are studying
In this paper, we study the MoDiv design for the TWRC under a simple AF scheme
and HARQ-CC protocol. We first derive an upper bound for the uncoded bit-error
rate (BER) of TWRC-AF channel under the Rayleigh fading condition, given $M$
different mapping schemes corresponding to each (re)transmission. Based on this
upper bound, we formulate a successive BER minimization MoDiv design into a
series of Quadratic Assignment Problem (QAP) in Koopmans-Beckmann (KB)
form~\cite{koopmans1957assignment}. Although QAP is NP-hard, efficient numerical
algorithms have been extensively researched~\cite{benlic2015memetic}, some of
which have shown extremely high performance over
QAPLIB~\cite{burkard1997qaplib}. We adopt a taboo search
algorithm~\cite{taillard1991robust} to solve each QAP in our formulation. Moreover, the coefficients of QAP problem can be also be computed efficiently in a
successive manner based on the solution to the preceding QAP problem. Our
numerical results demonstrate significant BER reduction over both non-MoDiv and
a simple heuristic MoDiv retransmission scheme for 16-QAM, 32-QAM and 64-QAM
constellation, even under mismatching design parameters.

% Structure of the manuscript
The paper is organized as follows. Section~\ref{sec:model} introduces the
TWRC-AF model and the HARQ protocol we are using. Section~\ref{sec:modiv}
presents the successive BER minimization MoDiv design problem. In
Section~\ref{sec:numerical}, we present the numerical results to show the
performance gain of our MoDiv scheme. Finally, Section~\ref{sec:conclusion}
concludes the paper.

\section{System Model}
\label{sec:model}
% Two-way relay channel and HARQ protocol

% Notations and the received signal model

% The ML demodulator equation

\section{Successive Constellation Mapping Design For Modulation Diversity}
\label{sec:modiv}
% Outline

\subsection{A BER upperbound}
\label{ssec:ber}
% The upper bound of BER using PEP

% Recursion of the approximated PEP

\subsection{The Successive Quadratic Assignment Problem}
\label{ssec:qap}
% BER minimization

% Approximated BER minimization

% Successively approximated BER minimization in SQAP form

% Solution via Tabu search


\section{Numerical Results}
\label{sec:numerical}
 % Simulation settings
 
 % Average uncoded BER and BER upperbounds of Non-MoDiv, Seddik, and QAP of for
 % (16) 32 (64) QAM vs d
 
 % Coded BER for 16, (32), 64 QAM for Non-MoDiv, Seddik, and QAP in one graph, M
 % = 3
 
  % Coded BER for 16, (32), 64 QAM for Non-MoDiv, Seddik, and QAP in one graph, M
 % = 4

\section{Conclusion}
\label{sec:conclusion}



% To start a new column (but not a new page) and help balance the last-page
% column length use \vfill\pagebreak.
% -------------------------------------------------------------------------
%\vfill
%\pagebreak

\vfill\pagebreak

% References should be produced using the bibtex program from suitable
% BiBTeX files (here: strings, refs, manuals). The IEEEbib.bst bibliography
% style file from IEEE produces unsorted bibliography list.
% -------------------------------------------------------------------------
\bibliographystyle{IEEEbib}
\bibliography{IEEEabrv,refs}

\end{document}
